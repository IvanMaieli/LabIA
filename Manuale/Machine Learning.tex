\documentclass[a4paper]{article}
\usepackage[utf8]{inputenc}


\title{Introduzione ai Fondamenti di Intelligenza Artificiale}
\author{Ivan Maieli}
\date{}

\begin{document}
\maketitle
\tableofcontents

\newpage

\section*{Prefazione}
\textit{
Saluti a tutti i lettori interessati. \\
Il presente documento è redatto al fine di presentare un'offerta di supporto specializzato per il workshop sull'intelligenza artificiale assegnatoci dai rappresentanti della scuola "E. Fermi" di Siracusa. \\
Il laboratorio si concentra su argomenti di base relativi all'intelligenza artificiale, proponendo un breve corso di durata approssimativa di due ore. \\
Durante questo intervallo temporale, sarà esaminato un algoritmo specifico, ritenuto ottimale per una prima introduzione all'argomento.\\
Con l'obiettivo di fornire una panoramica storica e teorica esaustiva, è stata inclusa un'ampia gamma di informazioni, comprese nozioni di cultura generale, miti legati alle intelligenze artificiali e le corrispondenti verità. \\
Questa scelta metodologica mira a fornire un quadro completo dell'evoluzione storica e del contesto teorico in cui si sviluppa l'intelligenza artificiale.\\
La partecipazione a questo workshop non richiede competenze specializzate, ma solo una comprensione di base di algebra, la familiarità con il concetto di funzione in ambito matematico e una cognizione essenziale sul funzionamento di uno script generico. \\
Questa scelta mira a rendere il contenuto accessibile a un pubblico eterogeneo, garantendo un'esperienza formativa inclusiva. \\
Il documento nasce dalla necessità di diffondere informazioni accurate sull'intelligenza artificiale, campo spesso soggetto a disinformazione e interpretazioni errate. \\
L'obiettivo è quello di fornire una base informativa solida, contrastando voci di corridoio e congetture fantascientifiche che possono influenzare erroneamente la percezione dell'intelligenza artificiale. \\
Mi auguro che i contenuti presentati in questo workshop siano di valore e di gradimento per gli interessati. L'obiettivo finale è quello di contribuire alla comprensione accurata e approfondita dell'intelligenza artificiale, superando fraintendimenti e promuovendo una visione basata su informazioni corrette e scientificamente fondate.
\begin{flushright}
Ivan Maieli
\end{flushright}
}

\section{Introduzione ai Fondamenti di Intelligenza Artificiale}
\subsection{Definizione di Intelligenza Artificiale}
\subsection{Storia dell'Intelligenza Artificiale}
\subsection{Amo i film di fantascienza, davvero}

\section{Un computer è molto più stupido di quanto pensiate}
\subsection{I dati sono davvero importanti}
\subsection{Introduzione alla terminologia}
\subsection{Supervised Learning}
\subsection{Unsupervised Learning}
\subsection{Reinforcement Learning}

\section{Gallina vecchia fa buon brodo: il Perceptron}
\subsection{Non è magia}
\subsection{Descrizione matematica di un problema semplice}
\subparagraph{Gestione degli Input}
\subparagraph{Funzioni di attivazione}
\subparagraph{Calcolo errore}
\subparagraph{Aggiornamento pesi}
\subsection{I limiti}
\subsection{Finalmente un po' di codice}

\end{document}


